%% LyX 2.0.6 created this file.  For more info, see http://www.lyx.org/.
%% Do not edit unless you really know what you are doing.
\documentclass{article}
\usepackage[latin9]{inputenc}
\setlength{\parskip}{\medskipamount}
\setlength{\parindent}{0pt}
\usepackage{amsmath}
\usepackage{amssymb}
\usepackage{esint}

\makeatletter
%%%%%%%%%%%%%%%%%%%%%%%%%%%%%% User specified LaTeX commands.

%
\usepackage{amsfonts}\usepackage{nopageno}%%%  The following few lines affect the margin sizes. 
\usepackage{pgfplots}
\pgfplotsset{compat=1.6}

\pgfplotsset{soldot/.style={color=blue,only marks,mark=*}} \pgfplotsset{holdot/.style={color=blue,fill=white,only marks,mark=*}}

\addtolength{\topmargin}{-.5in}
\setlength{\textwidth}{6in}       
\setlength{\oddsidemargin}{.25in}              
\setlength{\evensidemargin}{.25in}         
  
\setlength{\textheight}{9in}
\renewcommand{\baselinestretch}{1}
\reversemarginpar   
%
%

\makeatother

\begin{document}

\title{STA 6326 Homework 2 Solutions}


\author{Maksim Levental}


\date{\today}
\maketitle
\begin{enumerate}
\item [1.35]Let $Q(A)=P(A|B)$. Firstly 
\[
Q(A)=P(A|B)=\frac{P(A\cap B)}{P(B)}\ge0
\]
by the non-negativity of $P(\cdot)$ and the hypothesis that $P(B)>0$.
Secondly 
\[
Q(\Omega)=P(\Omega)=\frac{P(\Omega|B)}{P(B)}=\frac{P(B)}{P(B)}=1
\]
Finally assume $A_{i},A_{j}$ for all $i,j$ are pairwise disjoint.
Then 
\[
Q\Bigg(\bigcup_{i=1}^{\infty}A_{i}\Bigg)=P\Bigg(\bigcup_{i=1}^{\infty}A_{i}\Bigg|B\Bigg)=\frac{P\Bigg(\bigg(\bigcup_{i=1}^{\infty}A_{i}\bigg)\cap B\Bigg)}{P(B)}=\frac{P\bigg(\bigcup_{i=1}^{\infty}(A_{i}\cap B)\bigg)}{P(B)}
\]
and since $A_{i},\, A_{j}$ for all $i,j$ are pairwise disjoint $(A_{i}\cap B),\,(A_{j}\cap B)$
are also pairwise disjoint for all $i,j$ and by the countable additivity
of $P(\cdot)$
\[
\frac{P\bigg(\bigcup_{i=1}^{\infty}(A_{i}\cap B)\bigg)}{P(B)}=\frac{\sum_{i=1}^{\infty}P(A_{i}\cap B)}{P(B)}=\sum_{i=1}^{\infty}\frac{P\bigg(A_{i}\cap B)\bigg)}{P(B)}=\sum_{i=1}^{\infty}P(A_{i}|B)=\sum_{i=1}^{\infty}Q(A_{i})
\]

\item [1.38]

\begin{enumerate}
\item If $P(B)=1$ then $P(A|B)=\frac{P(A\cap B)}{1}$ but $P(A)=P(A\cap B)+P(A\cap B^{c})$
and since $A\cap B^{c}\subset B^{c}$ and $P(A\cap B^{c})\leq P(B^{c})=1-P(B)=0$
it's the case that $P(A)=P(A\cap B)$ so $P(A|B)=P(A)$.
\item $P(B|A)=P(B\cap A)/P(A)$ but the hypothesis $A\subset B$ implies
$B\cap A=A$ so $P(B|A)=P(A)/P(A)=1$. Then $P(A|B)=P(B|A)P(A)/P(B)=P(A)/P(B)$.
\item 
\begin{align*}
P(A|A\cup B)= & \frac{P\bigg(A\cap(A\cup B)\bigg)}{P(A\cup B)}\\
 & \frac{P(A)}{P(A)+P(B)-P(A\cap B)} & \text{by }A\subset A\cup B\\
 & \frac{P(A)}{P(A)+P(B)} & \text{by "mutually exclusive"}\iff A\cap B=\emptyset
\end{align*}

\item $P(A\cap B\cap C)=P(A|B\cap C)P(B\cap C)=P(A|B\cap C)P(B|C)P(C)$.
\end{enumerate}
\item [1.39]

\begin{enumerate}
\item If $P(A)>0$ and$P(B)>0$ and$P(A\cap B)=0$ then obviously $P(A)\cdot P(B)\ne P(A\cap B)$.
\item If $P(A)>0$ and$P(B)>0$ and $P(A)P(B)=P(A\cap B)$ then obviously
$P(A\cap B)=P(A)P(B)>0$.
\end{enumerate}
\item [1.44]The number of correct answers is binomially distributed with
$p=.25$ and $1-p=.75$. Then 
\[
P(X\ge10)=\sum_{k=10}^{20}\binom{20}{k}\bigg(\frac{1}{4}\bigg)^{k}\bigg(\frac{3}{4}\bigg)^{n-k}=.0138644
\]

\item [1.46]There $7^{7}=823,543$ different ways to distribute the 7 balls
into the 7 cells. The maximum number of cells that could have 3 balls
is 2 and clearly the minimum is 0. Hence $X_{3}\in\{0,1,2\}$. The
only way for $X_{3}=2$ would be 3 balls in a cell, 3 balls in another
cell, and the last ball in one cell. There are $\binom{7}{2}$ ways
to choose the 2 cells to have the 3 balls each, $\binom{7}{3}$ ways
to choose the first set of 3 balls for the first cell, $\binom{4}{3}$
to choose the second set of 3 balls for the second cell, then finally
$\binom{5}{1}=5$ different ways to choose which cell will contain
the last balls. Therefore 
\[
P(X_{3}=2)=\frac{\binom{7}{2}\binom{7}{3}\binom{4}{3}5}{7^{7}}\approx.0178
\]
For $X_{3}=1$ there are 3 different configurations possible:$\{3,1,1,1,1\},\{3,2,1,1\},\{3,2,2\}.$
\begin{align*}
 & \#\{3,1,1,1,1\} & =7\binom{7}{3}\times\binom{6}{4}\times4\times3\times2 & \text{ which cell contains 3 balls \ensuremath{\times}\ which 3 balls \ensuremath{\times}\ }\\
 &  &  & \text{ which cells contain 1 ball \ensuremath{\times}\ permute the balls}\\
 & \#\{3,2,1,1\} & =7\binom{7}{3}\times6\times\binom{4}{2}\times\binom{5}{2}\times2 & \text{ which cell contains 3 balls \ensuremath{\times}\ which 3 balls \ensuremath{\times}\ }\\
 &  &  & \text{ which cell contains 2 balls, which 2 balls,}\\
 &  &  & \text{ which cells contain 1 ball each, permute the balls}\\
 & \#\{3,2,2\} & =7\binom{7}{3}\times\binom{6}{2}\times\binom{4}{2} & \text{ which cell contains 3 balls, which 3 balls,}\\
 &  &  & \text{ which cells contain 2 balls, which two balls in the first 2-ball cell}\\
 &  &  & \text{ which cell contains second set of 2 balls, permute the balls}\\
 & \#\{3,4\} & =7\binom{7}{3}\times6 & \text{ which cell contains 3 balls, which 3 balls,}\\
 &  &  & \text{ which cell contains 4 balls}\\
 & +\\
 \hline
 &  & 288,120
\end{align*}
Hence $P(X_{3}=1)=1-288,120/7^{7}\approx.650146$. For $X_{3}=0$
there are very many configurations but we can compute by computing
as the complement of $P(X_{3}=1):$ $P(X_{3}=0)=1-P(X_{3}=1)+P(X_{3}=2)=1-0.178-0.650\approx.33$.
\item [1.47]Requirements for being a CDF: (i) right continuous (ii) $\lim_{x\rightarrow-\infty}F(x)=0$
and $\lim_{x\rightarrow\infty}F(x)=1$ (iii) non-decreasing.

\begin{enumerate}
\item $\frac{1}{2}+\frac{1}{\pi}\arctan(x)$

\begin{enumerate}
\item Continuous and hence right-continuous.
\item $\lim_{x\rightarrow-\pi/2}\tan(x)=-\infty$ hence $\lim_{x\rightarrow-\infty}\frac{1}{2}+\frac{1}{\pi}\arctan(x)=\frac{1}{2}+\frac{1}{\pi}\frac{-\pi}{2}=\frac{1}{2}-\frac{1}{2}=0$.
$\lim_{x\rightarrow\pi/2}\tan(x)=\infty$ hence $\lim_{x\rightarrow\infty}\frac{1}{2}+\frac{1}{\pi}\arctan(x)=\frac{1}{2}+\frac{1}{\pi}\frac{\pi}{2}=\frac{1}{2}+\frac{1}{2}=1$.
\item $\arctan(x)'=\frac{1}{1+x^{2}}>0$ is non-decreasing (monotonically
increasing) and hence $\frac{1}{2}+\frac{1}{\pi}\arctan(x)$ is non-decreasing.
\end{enumerate}
\item $(1+e^{-x})^{-1}$

\begin{enumerate}
\item Continuous and hence right-continuous.
\item $\lim_{x\rightarrow-\infty}e^{-x}=\infty$ hence $\lim_{x\rightarrow-\infty}(1+e^{-x})^{-1}=1/\infty=0$.
$\lim_{x\rightarrow\infty}e^{-x}=0$ hence $\lim_{x\rightarrow-\infty}(1+e^{-x})^{-1}=1/1=1$.
\item $\bigg((1+e^{-x})^{-1}\bigg)^{'}=-1(1+e^{-x})^{-2}(-1)e^{-x}=(1+e^{-x})e^{-x}>0$
hence non-decreasing.
\end{enumerate}
\item $\exp(-e^{-x})$

\begin{enumerate}
\item Continuous and hence right-continuous.
\item $\lim_{x\rightarrow-\infty}e^{-x}=\infty$ hence $\lim_{x\rightarrow-\infty}\exp(-e^{-x})=0$.
$\lim_{x\rightarrow\infty}e^{-x}=0$ hence $\lim_{x\rightarrow-\infty}\exp(-e^{-x})=1$.
\item $\bigg(\exp(-e^{-x})\bigg)^{'}=\exp(-e^{-x})\bigg(-e^{-x}\bigg)(-1)=\exp(-e^{-x})e^{-x}>0$
hence non-decreasing.
\end{enumerate}
\item $1-e^{-x}$

\begin{enumerate}
\item Continuous and hence right-continuous.
\item $e^{-0}=1$ hence $\lim_{x\rightarrow0}1-e^{-x}=1-1=0$. $\lim_{x\rightarrow\infty}e^{-x}=0$
hence $\lim_{x\rightarrow\infty}1-e^{-x}=1$. 
\item $\bigg(1-e^{-x}\bigg)^{'}=1+e^{-x}>0$ hence non-decreasing.
\end{enumerate}
\item If $0<\epsilon<1$ then $F_Y(y) = \left\{      
			\begin{array}{lr}        
			\frac{1-\epsilon}{1+e^{-y}} & \text{if } y < 0 \\        
			\epsilon + \frac{1-\epsilon}{1+e^{-y}} & \text{if } y \geq 0      
			\end{array}    
		\right. 
$ 

\begin{enumerate}
\item Since each piece of the piecewise definition of $F_{Y}$ is continuous
and the domain is defined with equality from the right ($y\geq0)$
$F_{Y}$ is right-continuous. 
\item $\lim_{y\rightarrow-\infty}\frac{1-\epsilon}{1+e^{-y}}=0$ similarly
to (b.ii) and $\lim_{y\rightarrow\infty}\frac{1-\epsilon}{1+e^{-y}}=1-\epsilon$
hence $\lim_{y\rightarrow\infty}\epsilon+\frac{1-\epsilon}{1+e^{-y}}=\epsilon+1-\epsilon=1$.
\item $\frac{1-\epsilon}{1+e^{-y}}$ and $\epsilon+\frac{1-\epsilon}{1+e^{-y}}$
are non-decreasing by (b.ii) and $\lim_{y\uparrow0}F_{Y}(y)=(1-\epsilon)/2<\epsilon+(1-\epsilon)/2=\lim_{y\downarrow0}F_{Y}(y)$
hence non-decreasing.
\end{enumerate}
\end{enumerate}
\item [1.49]Assume $F_{X}(t)\leq F_{Y}(t)$ for all $t$ and $F_{X}(t)<F_{Y}(t)$
for some $t_{0}$. By definition $F_{X}(t)=P(X\leq t)=1-P(X>t)$.
Similarly $F_{Y}(t)=P(Y\leq t)=1-P(Y>t)$. Then 
\[
1-P(X>t)\leq1-P(Y>t)
\]
 and so $P(X>t)\geq P(Y>t)$. Similarly for $t_{0}$ it's the case
$P(X>t_{0})>P(Y>t_{0})$.
\item [1.50]Let $S=\sum_{i=1}^{n}t^{k-1}$. Then
\[
S(1-t)=S-St=\sum_{i=1}^{n}t^{k-1}-\sum_{i=1}^{n}t^{k}=\sum_{i=1}^{n}(t^{k}-t^{k-1})
\]
But this last sum is telescoping hence 
\[
S(1-t)=1-t^{k-1}\implies S=\frac{1-t^{k-1}}{1-t}
\]

\item [1.51]$X$ is distributed Hypergeometrically, i.e. there are $\binom{30}{4}$
ways to draw a sample of 4 from 30, $\binom{5}{x}$ ways to draw $x$
microwaves from the subset of microwaves that is defective, and then
finally per draw from the defective subset there are $\binom{25}{4-x}$
to draw the remainder from the subset of functional microwaves. Hence
for $x=0,1,2,3,4$
\[
P(X=x)=\frac{\binom{5}{x}\binom{25}{4-x}}{\binom{30}{4}}
\]
So \[P(X=x) = \left\{      
			\begin{array}{lr}        
			\frac{2530}{5481} & \text{for } x=0 \\        
			\frac{2300}{5481} & \text{for } x=1 \\
			\frac{600}{5481} &  \text{for } x=2 \\
			\frac{50}{5481} &  \text{for } x=3 \\
			\frac{1}{5481} & \text{for } x=4 \\
			\end{array}    
		\right. 
\] Then \[F_X(x) = \left\{      
			\begin{array}{lr}        
			\frac{2530}{5481} & \text{for } x=0 \\        
			\frac{4830}{5481} & \text{for } x=1 \\
			\frac{5430}{5481} &  \text{for } x=2 \\
			\frac{5480}{5481} &  \text{for } x=3 \\
			\frac{5481}{5481} & \text{for } x=4 \\
			\end{array}    
		\right. 
\] The plot of the CDF is Figure \ref{fig:M1}.\begin{figure}
\centering
\begin{tikzpicture}
\begin{axis} 
\addplot[domain=0:1,blue] {0.461595}; 
\addplot[domain=1:2,blue] {0.881226}; 
\addplot[domain=2:3,blue] {0.990695};
\addplot[domain=3:4,blue] {0.999818};
\addplot[domain=4:5,blue] {1}; 
\addplot[holdot] coordinates{(1,0.461595)(2,0.881226)(3,0.990695)(4,0.999818)(5,1)}; 
\addplot[soldot] coordinates{(0,0.461595)(1,0.881226)(2,0.990695)(3,0.999818)(4,1)};
\end{axis}
\end{tikzpicture}
\caption{$F_X$ for problem 1.51 (note at $F_X(4)$ there is overlap) }\label{fig:M1}
\end{figure}
\item [1.52]Let $f(x)$ be a pdf with cdf $F(x)$, $F(x_{0})<1$, and \[g(x) = \left\{      
			\begin{array}{lr}        
			f(x)/(1-F(x_0)) & \text{if } x\geq x_0 \\        
			0 & \text{if } x < x_0      
			\end{array}    
		\right. 
\] Then since $f(x)\geq0$ and $1>F(x_{0})$ (and since $F$ is a cdf
$F(x_{0})\geq0$) it's the case that $g(x)\geq0$ for all $x$. Finally
\[
\int_{-\infty}^{\infty}g(x)=\int_{-\infty}^{x_{0}}g(x)+\int_{x_{0}}^{\infty}\frac{f(x)}{1-F(x_{0})}dx=0+\frac{1}{1-F(x_{0})}\int_{x_{0}}^{\infty}f(x)dx=\frac{1-\int_{_{-\infty}}^{x_{0}}f(x)}{1-F(x_{0})}=\frac{1-F(x_{0})}{1-F(x_{0})}=1
\]

\item [1.54]Let $c$ be a normalization constant.

\begin{enumerate}
\item $\int_{0}^{\pi/2}\sin(x)dx=1$ hence $c=1\implies\int_{0}^{\pi/2}cf(x)=1$.
\item $\int_{-\infty}^{\infty}e^{-|x|}dx=2$ hence $c=1/2\implies\int_{-\infty}^{\infty}ce^{-|x|}dx=1$. 
\end{enumerate}
\item [2.1]

\begin{enumerate}
\item If $Y=g(X)=X^{3}$ then $g^{-1}(Y)=\sqrt[3]{Y}$ and $\bigg(g^{-1}(Y)\bigg)^{'}=\frac{1}{3}y^{-2/3}$
and 
\[
f_{Y}(y)=42(\sqrt[3]{y})^{5}(1-\sqrt[3]{y})\bigg|\frac{1}{3}y^{-2/3}\bigg|=14(\sqrt[3]{y}-1)y
\]
Hence 
\[
\int_{0}^{1}f_{Y}(y)dy=14\int_{0}^{1}y(\sqrt[3]{y}-1)dy=14\frac{1}{14}=1
\]

\item If $Y=g(X)=4X+3$ then $g^{-1}(Y)=(Y-3)/4$ and $\bigg(g^{-1}(Y)\bigg)^{'}=\frac{1}{4}$
and
\[
f_{Y}(y)=7e^{-\frac{7}{4}(y-3)}\bigg|\frac{1}{4}\bigg|
\]
Hence, since $g(0)=4(0)+3<Y<\infty$
\[
\int_{0}^{\infty}f_{Y}(y)dy=\frac{7}{4}e^{\frac{21}{4}}\int_{3}^{\infty}e^{\frac{-7y}{4}}dy=e^{\frac{21}{4}}e^{-\frac{7\cdot3}{4}}=1
\]

\item If $Y=g(X)=X^{2}$ then $g(Y)^{-1}=\pm\sqrt{Y}$ and $\Bigg|\bigg(g^{-1}(Y)\bigg)^{'}\Bigg|=1/2\sqrt{Y}$
and
\[
f_{Y}(y)=\frac{30}{4}y^{2}\bigg(1-\frac{1}{2\sqrt{y}}\bigg)
\]
Hence
\[
\int_{0}^{1}f_{Y}(y)dy=\frac{30}{4}\int_{0}^{1}y^{2}\bigg(1-\frac{1}{2\sqrt{y}}\bigg)dy=\frac{30}{4}\frac{4}{30}=1
\]

\end{enumerate}
\item [2.2]

\begin{enumerate}
\item If $Y=g(X)=X^{2}$ then $g(Y)^{-1}=\pm\sqrt{Y}$ and $\Bigg|\bigg(g^{-1}(Y)\bigg)^{'}\Bigg|=1/2\sqrt{Y}$
and
\[
f_{Y}(y)=1\cdot\frac{1}{2\sqrt{y}}
\]

\item If $Y=g(X)=-\log(X)$ then $g^{-1}(Y)=e^{-Y}$ and $\Bigg|\bigg(g^{-1}(Y)\bigg)\Bigg|^{'}=e^{-Y}$
and
\[
f_{Y}(y)=\binom{n+m+1}{n,m,1}e^{-ny}(1-e^{-y})^{m}e^{-y}
\]
With domain $-\log(1)=0<y<-\log(0)=\infty$.
\item If $Y=g(X)=e^{X}$ then $g(Y)^{-1}=\log Y$ and $\Bigg|\bigg(g^{-1}(Y)\bigg)^{'}\Bigg|=1/Y$
and
\[
f_{Y}(y)=\frac{1}{\sigma^{2}}\frac{1}{y^{2}}e^{-(1/y\sigma)^{2}/2}=\frac{1}{(\sigma y)^{2}}e^{-(1/y\sigma)^{2}/2}
\]

\end{enumerate}
\item [2.3]If $Y=g(X)=X/(X+1)$ then $g(Y)^{-1}=1/(1-Y)$ and $\Bigg|\bigg(\frac{1}{1-y}\bigg)^{'}\Bigg|=\frac{1}{(1-y)^{2}}$
Hence for $y=0,\frac{1}{2},\frac{2}{3},\frac{3}{4},...$ 
\[
f_{Y}(y)=\frac{1}{3}\bigg(\frac{2}{3}\bigg)^{\frac{1}{1-y}}\frac{1}{(1-y)^{2}}
\]
\\
\\
\\

\item [2.4]$f(x) = \left\{      
			\begin{array}{lr}        
			\frac{1}{2}\lambda e^{-\lambda x} & \text{if } x \geq 0 \\        
			\frac{1}{2}\lambda e^{\lambda x} & \text{if } x < 0      
			\end{array}    
		\right. 
$ 

\begin{enumerate}
\item $e^{\lambda x}>0$ for all $x\in(-\infty,\infty)$ hence $f(x)\geq0$.
Furthermore 
\[
\int_{-\infty}^{\infty}f(x)dx=\frac{\lambda}{2}\int_{-\infty}^{0}e^{\lambda x}dx+\frac{\lambda}{2}\int_{0}^{\infty}e^{-\lambda x}dx
\]
Then by $-u=x$ 
\[
\frac{\lambda}{2}\int_{-\infty}^{0}e^{\lambda x}dx=\frac{\lambda}{2}\int_{\infty}^{0}e^{-\lambda u}d(-u)=(-1)(-1)\frac{\lambda}{2}\int_{0}^{\infty}e^{-\lambda u}du=\frac{\lambda}{2}\int_{0}^{\infty}e^{-\lambda x}dx
\]
and hence
\[
\int_{-\infty}^{\infty}f(x)dx=2\frac{\lambda}{2}\int_{0}^{\infty}e^{-\lambda x}dx=\lambda\frac{1}{\lambda}(1-0)=1.
\]

\item If $x<0$ then
\[
F_{X}(x)=\frac{\lambda}{2}\int_{-\infty}^{x}e^{\lambda x}dx=\frac{\lambda}{2}\frac{1}{\lambda}e^{\lambda x}=\frac{1}{2}e^{\lambda x}
\]
If $x\geq0$ then 
\[
F_{X}(x)=\frac{\lambda}{2}\int_{-\infty}^{0}e^{\lambda x}dx+\frac{\lambda}{2}\int_{0}^{x}e^{-\lambda x}dx=\frac{1}{2}+\bigg(\frac{1}{2}-\frac{1}{2}e^{-\lambda x}\bigg)
\]

\item $P(|X|<t)=\int_{-t}^{t}f(x)dx.$ Arguments from part (a) imply 
\[
\int_{-t}^{t}f(x)dx=2\frac{\lambda}{2}\int_{0}^{t}e^{-\lambda x}dx=2\bigg(\frac{1}{2}-\frac{1}{2}e^{-\lambda t}\bigg)
\]
 \end{enumerate}
\end{enumerate}

\end{document}
