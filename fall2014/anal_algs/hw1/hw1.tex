%% LyX 2.0.6 created this file.  For more info, see http://www.lyx.org/.
%% Do not edit unless you really know what you are doing.
\documentclass{article}
\usepackage[latin9]{inputenc}
\setlength{\parskip}{\medskipamount}
\setlength{\parindent}{0pt}
\usepackage{amsmath}
\usepackage{amssymb}

\makeatletter
%%%%%%%%%%%%%%%%%%%%%%%%%%%%%% User specified LaTeX commands.

%
\usepackage{amsfonts}\usepackage{nopageno}%%%  The following few lines affect the margin sizes. 
\addtolength{\topmargin}{-.5in}
\setlength{\textwidth}{6in}       
\setlength{\oddsidemargin}{.25in}              
\setlength{\evensidemargin}{.25in}         
  
\setlength{\textheight}{9in}
\renewcommand{\baselinestretch}{1}
\reversemarginpar   
%
%

\makeatother

\begin{document}

\title{COT5405 Homework 1 Solutions}


\author{Maksim Levental}


\date{\today}
\maketitle
\begin{enumerate}
\item [2.4]

\begin{enumerate}
\item (2,1), (3,1), (8,1), (6,1), (8,6)
\item Reverse sorted, i.e. $\left\{ n,n-1,\cdots,1\right\} $. The number
of inversions is $n-1$ for 1 because there are $n-1$ elements in
the array which are larger than 1 but preceed it in the array, $n-2$
for 2 because there $n-2$ elements which are larger than 2 (exceptions
are 2 and 1) but preceed it, and so on. So for $i=1,2,\cdots,n$ the
number of inversions induced is $n-i$ . In sum $\sum_{i=1}^{n}n-i=n\sum_{i=1}^{n}1-\sum_{i=1}^{n}i=n^{2}-\frac{{n(n+1)}}{2}=n^{2}-\frac{{(n^{2}+n)}}{2}=\frac{{n(n-1)}}{2}$
\item Call an inversion of type $(j,i)$ induced by an element $A[j]$ if
$j$ is such that $A[j]<A[i]$ and $i<j$ and let $|(j,i)|$ be the
number of such inversions. Then $|(j,i)|$ is the number of swaps
that will have to be performed on element $A[j]$ before it is in
its proper position. To see that this is the case note that all $(j,i)$
inversions persist through the sorting process, up until $A[j]$ is
sorted, since for all $i<j$ element $A[i]$ will be inserted into
the sorted portion of the array prior to $A[j].$ Therefore upon inserting
$A[j]$ there will still be $|(j,i)|$ inversions and therefore $|(j,i)|$
swaps. Consequently $\sum_{j}|(j,i)|$ the total number of inversions
in the array is the total number of swaps performed by insertion sort,
i.e. directly proportional insertion sort's running time.
\item Modify merge sort such that when function returns from the two recursive
calls, when the ``merging'' is done, it counts the number of elements
in the ``left'' array each time an element is chosen from the front
of the ``right'' array ( after comparison between the leading elements
of both arrays). The quantity of elements in the ``left'' array
each time an element from the front of the right array is chosen is
by definition the number of elements in the original array that were
greater than that chosen element and yet preceeded it. Furthermore
there is no double counting because once a merge happens 2 elements
in the merge array are never compared again. Take for example the
array $\mbox{\ensuremath{\left[8,7,6,5,4,3,2,1\right]}}$ and suppose
the recursion bottoms out at 4 elements. Then the first recursion
returns $\left[5,6,7,8\right]$ and $\left[1,2,3,4\right]$. The merging
then selects each of the four elements from the ``right'' array
$\left[1,2,3,4\right]$ since each of the elements in the ``left''
array $\left[5,6,7,8\right]$ is greater than each in the ``right''.
Manifestly there are 4 inversion per element in the ``right'' array
- 1 for each element in the ``left'' array, and so the total is
16.
\end{enumerate}
\item [3.2]

\begin{enumerate}
\item []
\begin{align*}
A & B & O & o & \Omega & \omega & \Theta\\
\text{lg}^{k}n & n^{\epsilon} & yes & yes & yes & yes & yes\\
n^{k} & c^{n} & yes & yes & yes & yes & yes\\
\sqrt{n} & n^{\text{sin}n} & yes & yes & yes & yes & yes\\
2^{n} & 2^{n/2} & yes & yes & yes & yes & yes\\
n^{\text{lg}c} & c^{\text{lg}n} & yes & yes & yes & yes & yes\\
\text{lg}(n!) & \text{lg}(n^{n}) & yes & yes & yes & yes & yes
\end{align*}
\end{enumerate}
\end{enumerate}

\end{document}
