%% LyX 2.0.6 created this file.  For more info, see http://www.lyx.org/.
%% Do not edit unless you really know what you are doing.
\documentclass{article}
\usepackage[latin9]{inputenc}
\usepackage{listings}
\lstset{frame=single}
\setlength{\parskip}{\medskipamount}
\setlength{\parindent}{0pt}
\usepackage{amsmath}
\usepackage{amssymb}

\makeatletter
%%%%%%%%%%%%%%%%%%%%%%%%%%%%%% User specified LaTeX commands.

%
\usepackage{mathdots}
\usepackage{listings}
\usepackage{tikz}
\usepackage{pgf}
\usepackage{tikz-qtree}
\usepackage{multirow}
\usepackage{rotating}
\usepackage{amsfonts}\usepackage{nopageno}%%%  The following few lines affect the margin sizes. 
\addtolength{\topmargin}{-.5in}
\setlength{\textwidth}{6in}       
\setlength{\oddsidemargin}{.25in}              
\setlength{\evensidemargin}{.25in}         
  
\setlength{\textheight}{9in}
\renewcommand{\baselinestretch}{1}
\reversemarginpar   

\lstset{
numbers=left
}


%
%

\makeatother

\begin{document}

\title{COT5405 Homework 1 Solutions}


\author{Maksim Levental}


\date{\today}
\maketitle






I'm not good at Algebra. Supposedly there are two camps in math, the geometers and the algebraists, and never the twain shall meet. 
I'm loathe to affirm such a pseudo-scientific hypothesis but I simply cannot reason about purely algebraic ideas as well as I can about 
even slightly geometric forms. Case in point: I do not have an intuitive understanding of the Euclidean Algorithm for computing the GCD 
of two numbers. Sure I understand the proof perfectly well and it's short enough that I can hold it all in short term memory when I'm 
reading it, but I have trouble recreating it because there's no kernel that's manifestly commonsensical to me. Juxtapose that idea with, 
for example, Gram-Schmidt orthonormalization. If I never do math again I'd still be able to construct an orthonormal basis for a vector 
space on my death bed, with my eyes closed, and just after having my hands amputated.
 
Euclidean algorithm to the rescue (or maybe Euclid to the rescue). Look at this image of a rectangle that's 24 by 60 units.
 
 The idea is that finding the GCD of 24 and 60, from here on denoted by $\text{gcd}(24,60)$,
 corresponds to tiling the rectangle with the largest squares of uniform area possible, i.e. some number that all have the same area, 
 with the characteristic dimension of the tiling square being that GCD. Why is this the case? Well firstly, for such a tiling is the 
 characteristic dimension of the tiling square even a common divisor at all? Sure. Any row of the tiling evenly divides the width 
 dimension of the rectangle and any column of the tiling divides the height dimension of the rectangle. In this instance
 (in the above image) it's a little confusing but the tiling squares are the ones with embossed edges - the tiling is 2x4
 with squares of characteristic dimension 12. So $24 \div 2 = 12$ and $60 \div 4 = 12$ and so 12 is a common divisor. Why 
 is this common divisor the <strong>greatest</strong> common divisor? That takes a little more work, and admittedly requires
 a fact that's not so geometric, but simple enough to prove that I'm comfortable with it anyway.
 
 The deal is that the tiling 
 above was, presumably (or potentially), derived using the full Euclidean algorithm for computing greatest common divisors. 
 Here's an illustration of the geometric representation of said algorithm (albeit on a rectangle with dimension 462x1071):
 
 What's happening here? The tiling proceeds by attempting to tile the rectangle by the largest possible square that's possibly a
 divisor of both 462 and 1071, namely a square with characteristic dimension 462. Clearly it falls short - there's a remainder of
 $1071-2\cdot 462=147$. Why then does the tiling proceed with the blue squares (then on down to the red ones)? Here's the part 
 that's purely algebraic (at least as far as I can tell) and justification for iteration with smaller squares: I claim that

 $\text{gcd}(a,b) = \text{gcd}(a,b-at)$ for any $t \in \mathbb{Z}$
 
 Proof:
 
 Let $d=gcd(a,b)$. Then $d$ divides $a$ and $b$, and then $d$ divides $at $ (you get at least one factor of $d$ from $a$).
 So $d$ divides $b-at$ ($d$ "factors" out) and therefore $d \leq \text{gcd}(a,b-at)$. Let $d'=\text{gcd}(a,b-at)$. Then $d'$ divides
 $a$ and $b-at$ and so $d'$ divides $(b-at)+at$, i.e. $d'$ divides $b$. Therefore $d' \leq \text{gcd}(a,b)$. Finally $d=d'$.
 Bringing it back - what does this small fact do for us? Look at the gif. If tiling the larger rectangle, suppose without having 
 to resort to iterations, by largest area possible squares is equivalent to computing the GCD of the dimensions of the rectangle,
 then tiling the remainder with largest possible squares (recursing) computes the GCD of the dimensions of the remainder, by the 
 above fact, therefore the larger rectangle. To see this let's compute the characteristic dimensions of all the different color 
 squares in the gif (the rectangle has dimensions 462 x 1071 as mentioned before): 
 $\text{orange} = 462$, $\text{blue}= 1071-2\cdot \text{orange} = 147$, $\text{red} = \text{orange}-3\cdot \text{blue} = 462-3\cdot 147 = 21$. 
 Then the fact about "recursive" GCDs enables:
 \begin{align*} \text{gcd}(462,1071) = \text{gcd}(\text{orange},1071)\\
 =\text{gcd}(\text{blue},\text{orange})\\
 =\text{gcd}(\text{red},
 \text{blue})\\
 =\text{gcd}(0,\text{red})\\
 =\text{red}\\
 = 21
 \end{align*}
 The way that Euclid himself describes it is finding a common "measure" of two lengths (i.e. greatest common integer length) by 
 successively measuring by remainders.
 
 Notice that you can represent $\text{gcd}(462,1071)$ as a linear combination of $462$ 
 and $1071$:
 
 \begin{align*}
 147 &=1071-2\cdot 462 & && 21&= 462-3(1071-2\cdot 462) \\
 21 &=462-3\cdot 147 & \implies && &= 462 -3\cdot 1071 + 6\cdot 462 \\
 0 &= 147-7 \cdot 21 & && &= 7\cdot 462 -3\cdot 1071
 \end{align*} 
 This is true in general and called Bézout's identity (in fact smallest such linear combination is another characterization
 of $\text{gcd}(a,b)~$ ). It'll be useful in a future post for computing inverses of elements in $\mathbb{Z}_p^*$ for $p$ prime.
 How fast does Euclid's algorithm run? Interestingly enough, but maybe not surprisingly, it's related to the Fibonacci sequence .
 I think this is because of the "measuring common lengths" thing and that the Fibonacci sequence is related to partitioning a length 
 such that the ratio of the two partitions is optimal, in some sense.
 
 Lamé's Theorem:
 For any integer $k \geq 1$, if $a &gt; b \geq 1$ and $b &lt; F_{k+1}$, where $F_i$ is the ith Fibonacci number, Euclid's 
 algorithm computes $\text{gcd}(a,b)$ in fewer than $k$ recursive calls to itself.
 The implication, since $b &lt; F_{k_1} \approx \phi^k / \sqrt{5}$, is that $k = O(\lg b)$.

 






















\end{document}
