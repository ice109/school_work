% --------------------------------------------------------------
% This is all preamble stuff that you don't have to worry about.
% Head down to where it says "Start here"
% --------------------------------------------------------------
 
\documentclass[12pt]{article}
 
\usepackage[margin=1in]{geometry} 
\usepackage{amsmath,amsthm,amssymb}
\usepackage{graphicx}
\newcommand{\N}{\mathbb{N}}
\newcommand{\Z}{\mathbb{Z}}
\setlength\parindent{0pt}
\newenvironment{theorem}[2][Theorem]{\begin{trivlist}
\item[\hskip \labelsep {\bfseries #1}\hskip \labelsep {\bfseries #2.}]}{\end{trivlist}}
\newenvironment{lemma}[2][Lemma]{\begin{trivlist}
\item[\hskip \labelsep {\bfseries #1}\hskip \labelsep {\bfseries #2.}]}{\end{trivlist}}
\newenvironment{exercise}[2][Exercise]{\begin{trivlist}
\item[\hskip \labelsep {\bfseries #1}\hskip \labelsep {\bfseries #2.}]}{\end{trivlist}}
\newenvironment{problem}[2][Problem]{\begin{trivlist}
\item[\hskip \labelsep {\bfseries #1}\hskip \labelsep {\bfseries #2.}]}{\end{trivlist}}
\newenvironment{question}[2][Question]{\begin{trivlist}
\item[\hskip \labelsep {\bfseries #1}\hskip \labelsep {\bfseries #2.}]}{\end{trivlist}}
\newenvironment{corollary}[2][Corollary]{\begin{trivlist}
\item[\hskip \labelsep {\bfseries #1}\hskip \labelsep {\bfseries #2.}]}{\end{trivlist}}
\newenvironment{solution}
  {\begin{proof}[Solution]\renewcommand{\qedsymbol}{}}
  {\end{proof}}

\usepackage{listings}
\usepackage{color}

\definecolor{mygreen}{rgb}{0,0.6,0}
\definecolor{mygray}{rgb}{0.5,0.5,0.5}
\definecolor{mymauve}{rgb}{0.58,0,0.82}

\lstset{ %
  backgroundcolor=\color{white},   % choose the background color; you must add \usepackage{color} or \usepackage{xcolor}
  basicstyle=\footnotesize,        % the size of the fonts that are used for the code
  breakatwhitespace=false,         % sets if automatic breaks should only happen at whitespace
  breaklines=true,                 % sets automatic line breaking
  captionpos=b,                    % sets the caption-position to bottom
  commentstyle=\color{mygreen},    % comment style
  deletekeywords={...},            % if you want to delete keywords from the given language
  escapeinside={\%*}{*)},          % if you want to add LaTeX within your code
  extendedchars=true,              % lets you use non-ASCII characters; for 8-bits encodings only, does not work with UTF-8
  frame=single,                    % adds a frame around the code
  keepspaces=true,                 % keeps spaces in text, useful for keeping indentation of code (possibly needs columns=flexible)
  keywordstyle=\color{blue},       % keyword style
  language=Mathematica,                 % the language of the code
  morekeywords={*,...},            % if you want to add more keywords to the set
  numbers=left,                    % where to put the line-numbers; possible values are (none, left, right)
  numbersep=5pt,                   % how far the line-numbers are from the code
  numberstyle=\tiny\color{mygray}, % the style that is used for the line-numbers
  rulecolor=\color{black},         % if not set, the frame-color may be changed on line-breaks within not-black text (e.g. comments (green here))
  showspaces=false,                % show spaces everywhere adding particular underscores; it overrides 'showstringspaces'
  showstringspaces=false,          % underline spaces within strings only
  showtabs=false,                  % show tabs within strings adding particular underscores
  stepnumber=1,                    % the step between two line-numbers. If it's 1, each line will be numbered
  stringstyle=\color{mymauve},     % string literal style
  tabsize=2,                       % sets default tabsize to 2 spaces
  title=\lstname                   % show the filename of files included with \lstinputlisting; also try caption instead of title
}

\begin{document}


 
% --------------------------------------------------------------
%                         Start here
% --------------------------------------------------------------
\title{notes}
\maketitle

\date{4/9/2014}\\

The first isomorphism theorem for groups is:\\

Let G and H be groups, and let $\phi:G\rightarrow H$ be a homomorphism. Then the kernel of $\phi$ is a normal subgroup of $G$, the image of $\phi$ is a subgroup of $H$, and the image of $\phi$ is isomorphic to the quotient group $G/\ker(\phi)$ 

\begin{proof}
Let $K = \ker(\phi)$ and $\tilde{\phi}: G/K \rightarrow \text{im}(\phi)$ be defined in terms of the right cosets of $K$, i.e. $gK \mapsto \phi(g)$. Then the $\tilde{\phi}$ is well-defined, i.e. different representatives of the same coset map to the same element, because if $g'K = gK$ then $g'=gk$ for some $k \in K$ and so 

$$\phi(g') = \phi(gk) = \phi(g)\phi(k) = \phi(g)\cdot e = \phi(g)$$.

$\tilde{\phi}$ is a homomorphism because $\phi$ is:

\begin{align*}
\tilde{\phi}(gK g'K) &= \tilde{\phi}(gg'K) & \text{by coset multiplication} \\
&= \phi(gg') & \text{by definition of $\tilde{\phi}$} \\
&= \phi(g)\phi(g') & \text{$\phi$ is a homomorphism} \\
&= \tilde{\phi}(gK)\tilde{\phi}(g'K) & \text{by definition of $\tilde{\phi}$} 
\end{align*}

$\tilde{\phi}$ is injective because $\ker(\tilde{\phi})$ is only $K$, which is the 0 in $G/\ker(\phi)$: If $\tilde{\phi}(gK) = e$ then $\phi(g) = e$ and so $g \in \ker(\phi)$ making $gK = K$. And $\tilde{\phi}$ is surjective because $ \text{im}(\tilde{\phi}) = \text{im}(\phi)$.

\end{proof}

A good example is the rank-nullity theorem from Linear Algebra. It's stated:

If $A$ is a matrix with $m$ rows and $n$ columns (it maps from $R^n$ to $R^m$ because $m$ dot products with the $n$ length column vector are carried out) then the dimensions of the column space and the row space of $A$ are the same, collectively called the rank $r$, and the dimension of the nullspace is $n-r$ and the dimension of the nullspace of $A^T$ is $m-r$.

\begin{proof}
Let $T$ be the homomorphism that corresponds to $A$, that maps between $V=R^n$ and $W=R^m$. The theorem says that $V/\text{ker($T$)}$
is isomorphic to $\text{im}(T)$. But im$(T)$ is of course the column space and $V/\text{ker}(T)$ is (the only thing i can say so far is that the basis for this space maps back to a basis for $V$ that is linearly independent from the basis for the kernel. need to think about dual space to figure out orthogonal complement and see if the algebra will give me that too - but i suspect the algebra is agnostic to orthogonality without incorporating inner products, which is exactly what the dual space will do for me)
\end{proof}

\date{5/29/2014}\\

Fibonacci numbers and generating functions (CLRS 4-4): \\

Let $\mathcal{F}(z) = \sum_{i=0}^\infty F_i z^i$ where $F_i$ is the $i$th Fibonacci number. \\

Prove that $\mathcal{F}(z) = z + z \mathcal{F}(z) + z^2 \mathcal{F}(z)$.

\begin{proof}
You can think of the equation as a recurrence relation; multiplying by $z$ and $z^2$ ``aligns'' $i$th and $i+1$th coefficients and then you sum them.
So 

\begin{align*} 
z \mathcal{F}(z) &= z^2 + z^3 + 2z^4 + 3z^5 + 5z^6 +8z^7 + 13z^8 + 21z^9 \\
z^2 \mathcal{F}(z) &=0 + z^3 + z^4 + 2z^5 + 3z^6 + 5z^7 +8z^8 + 13z^9 + 21z^{10}
\end{align*}

\end{proof}

Show that 
$$
\mathcal{F}(z) = \frac{1}{\sqrt{5}}\bigg(\frac{1}{1-\phi z}-\frac{1}{1-\hat{\phi}z}\bigg)
$$

where $\phi = \frac{1+\sqrt{5}}{2}$ and $\hat{\phi} = \frac{1-\sqrt{5}}{2}$.

\begin{proof}
Using just prior results 

\begin{align*}
\mathcal{F}(z) &= \frac{z}{1-z-z^2} \\
&= \frac{z}{(1-\phi z)(1-\hat{\phi}z)} \\
&=  \frac{1}{\sqrt{5}}\bigg(\frac{1}{1-\phi z}-\frac{1}{1-\hat{\phi}z}\bigg)
\end{align*}
\end{proof}

Show that 

$$
\mathcal{F}(z) = \sum_{i=0}^\infty \frac{1}{\sqrt{5}}(\phi^i - \hat{\phi}^i)z^i
$$

\begin{proof}
It's basically asking you to recompute the coefficients of the generating function. You can do this from the analytic formula for $\mathcal{F}(z)$ derived just prior.
The trick is to compute derivatives of $\mathcal{F}(z)$ at $z=0$. Think about it. For example evaluate both the analytic formula and the generating function at $z=0$.
What do you get? Well first of all the agree but more importantly you get the constant coefficient in the power-series/generating function (which happens to be $0$).
Then take the first derivative of the power-series/generating function and evaluate it at 0. What do you get? You get the only term in left in the differentiated power-series
that doesn't have a $z$ factor, which will be $1!$ times the coefficient of that term. Now take the second derivative and evaluate at $z=0$. What do you get? The coefficient
of the second term in the power-series multiplied by $2!$. And so on. In general you terms of the form

$$
\frac{1}{\sqrt{5}}(i!\phi^i-i!\hat{\phi}^i)
$$

So just dividing by $i!$ gives you the $i$th coefficient of the power-series/generating function, i.e. $\frac{1}{\sqrt{5}}(\phi^i-\hat{\phi}^i)$.
The implication being that a closed form for the $i$th Fibonacci number is $\frac{1}{\sqrt{5}}(\phi^i-\hat{\phi}^i)$. 
Notice that for large $i$, since $|\hat{\phi}| < 1$, the $i$th Fibonacci number is just $\frac{1}{\sqrt{5}}(\phi^i)$. 

\end{proof}

\end{document}
