% --------------------------------------------------------------
% This is all preamble stuff that you don't have to worry about.
% Head down to where it says "Start here"
% --------------------------------------------------------------
 
\documentclass[12pt]{article}
 
\usepackage[margin=1in]{geometry} 
\usepackage{amsmath,amsthm,amssymb}
\usepackage{graphicx}

\newcommand{\N}{\mathbb{N}}
\newcommand{\Z}{\mathbb{Z}}
 
\newenvironment{theorem}[2][Theorem]{\begin{trivlist}
\item[\hskip \labelsep {\bfseries #1}\hskip \labelsep {\bfseries #2.}]}{\end{trivlist}}
\newenvironment{lemma}[2][Lemma]{\begin{trivlist}
\item[\hskip \labelsep {\bfseries #1}\hskip \labelsep {\bfseries #2.}]}{\end{trivlist}}
\newenvironment{exercise}[2][Exercise]{\begin{trivlist}
\item[\hskip \labelsep {\bfseries #1}\hskip \labelsep {\bfseries #2.}]}{\end{trivlist}}
\newenvironment{problem}[2][Problem]{\begin{trivlist}
\item[\hskip \labelsep {\bfseries #1}\hskip \labelsep {\bfseries #2.}]}{\end{trivlist}}
\newenvironment{question}[2][Question]{\begin{trivlist}
\item[\hskip \labelsep {\bfseries #1}\hskip \labelsep {\bfseries #2.}]}{\end{trivlist}}
\newenvironment{corollary}[2][Corollary]{\begin{trivlist}
\item[\hskip \labelsep {\bfseries #1}\hskip \labelsep {\bfseries #2.}]}{\end{trivlist}}
\newenvironment{solution}
  {\begin{proof}[Solution]\renewcommand{\qedsymbol}{}}
  {\end{proof}}

\usepackage{listings}
\usepackage{color}

\definecolor{mygreen}{rgb}{0,0.6,0}
\definecolor{mygray}{rgb}{0.5,0.5,0.5}
\definecolor{mymauve}{rgb}{0.58,0,0.82}

\lstset{ %
  backgroundcolor=\color{white},   % choose the background color; you must add \usepackage{color} or \usepackage{xcolor}
  basicstyle=\footnotesize,        % the size of the fonts that are used for the code
  breakatwhitespace=false,         % sets if automatic breaks should only happen at whitespace
  breaklines=true,                 % sets automatic line breaking
  captionpos=b,                    % sets the caption-position to bottom
  commentstyle=\color{mygreen},    % comment style
  deletekeywords={...},            % if you want to delete keywords from the given language
  escapeinside={\%*}{*)},          % if you want to add LaTeX within your code
  extendedchars=true,              % lets you use non-ASCII characters; for 8-bits encodings only, does not work with UTF-8
  frame=single,                    % adds a frame around the code
  keepspaces=true,                 % keeps spaces in text, useful for keeping indentation of code (possibly needs columns=flexible)
  keywordstyle=\color{blue},       % keyword style
  language=Mathematica,                 % the language of the code
  morekeywords={*,...},            % if you want to add more keywords to the set
  numbers=left,                    % where to put the line-numbers; possible values are (none, left, right)
  numbersep=5pt,                   % how far the line-numbers are from the code
  numberstyle=\tiny\color{mygray}, % the style that is used for the line-numbers
  rulecolor=\color{black},         % if not set, the frame-color may be changed on line-breaks within not-black text (e.g. comments (green here))
  showspaces=false,                % show spaces everywhere adding particular underscores; it overrides 'showstringspaces'
  showstringspaces=false,          % underline spaces within strings only
  showtabs=false,                  % show tabs within strings adding particular underscores
  stepnumber=1,                    % the step between two line-numbers. If it's 1, each line will be numbered
  stringstyle=\color{mymauve},     % string literal style
  tabsize=2,                       % sets default tabsize to 2 spaces
  title=\lstname                   % show the filename of files included with \lstinputlisting; also try caption instead of title
}

\begin{document}


 
% --------------------------------------------------------------
%                         Start here
% --------------------------------------------------------------
 
\title{Homework 1}%replace X with the appropriate number
\author{Maksim Levental\\ %replace with your name
MAP 4102} %if necessary, replace with your course title
 
\maketitle
 
\begin{problem}{2.31(a)} %You can use theorem, exercise, problem, or question here.  Modify x.yz to be whatever number you are proving
Let $T \sim Exp(\lambda)$. Compute $E(T|T<c)$.
\end{problem}
 
\begin{solution}\ \\

\noindent By definition 

$$
E(T|T<c) = \int_0^c t \mathbb{P}(T=t|T<c)dt
$$

\noindent But $\mathbb{P}(T=t|T<c)$ is the truncated distribution whose probability density function $f_T$ is 

$$
f_T(t) = \frac{g_T(t)}{F_T(c)}
$$

\noindent where $g_T$ is the probability density function for $T$, $\lambda e^{-\lambda t}$
, and $F_T$ is the cumulative distribution function for $T$, $1-\lambda e^{-\lambda t}$. Hence

\begin{align*}
E(T|T<c) &= \int_0^c t \mathbb{P}(T=t|T<c)dt =  \int_0^c t \frac{\lambda e^{-\lambda t}}{1-\lambda e^{-\lambda c}}dt\\
&= \frac{1}{\lambda} + \frac{c}{1-e^{c\lambda}}
\end{align*}


\end{solution}

\begin{problem}{2.31(b)} %You can use theorem, exercise, problem, or question here.  Modify x.yz to be whatever number you are proving
Let $T \sim Exp(\lambda)$. Compute $E(T|T<c)$.
\end{problem}
\begin{solution}\ \\

\noindent From $ET = \mathbb{P}(T<c)E(T|T<c) + \mathbb{P}(T>c)E(T|T>c)$ we have 

$$
E(T|T<c) = \frac{ET - \mathbb{P}(T>c)E(T|T>c)}{\mathbb{P}(T<c)}
$$

\noindent but 

$$
E(T|T>c) = \int_0^cdt + \int_c^{\infty}\mathbb{P}(T>t|T>c)dt = c+\int_c^{\infty} \mathbb{P}(T>t-c)dt
$$

\noindent by the tail formula for expectation and the memoryless property of the exponential distribution. Finally 

$$\int_c^{\infty} \mathbb{P}(T>t-c)dt = \int_c^{\infty}e^{-\lambda(t-c)}dt = \frac{1}{\lambda} $$

\noindent Hence 

\begin{align*}
E(T|T<c) &= \frac{\frac{1}{\lambda} - e^{-\lambda c}\big(\frac{1}{\lambda}+c\big)}{1-e^{-\lambda c}}\\
&= \frac{1}{\lambda} - \frac{e^{-\lambda c}}{1-e^{\lambda c}}c = \frac{1}{\lambda} + \frac{c}{1-e^{\lambda c}}
\end{align*}




\end{solution}

\begin{problem}{2.33} %You can use theorem, exercise, problem, or question here.  Modify x.yz to be whatever number you are proving

Suppose traffic on a road is accurately modeled by a Poisson process with rate parameter $\lambda\frac{cars}{minute}$ and a chicken needs $c$ minutes to cross the road. Show that the expected time for the chicken to cross, including wait, is $(e^{\lambda c}-1)/\lambda$. 
\end{problem}
 
\begin{solution}\ \\ 

\noindent Let $t_i$ be times between passages of cars and $J = \text{min}\{j: t_j > c\}$. Then $t_i$ is exponentially distributed for all $i$, and $J$ is geometrically distributed with success probability $p =\mathbb{P}(T|T>c) = e^{-\lambda c}$ and failure probability $1-p=\mathbb{P}(T|T<c) = 1 - e^{-\lambda c}$. The expectation value of $J$ is 
$$
\frac{1}{1-p} = \frac{1}{e^{-\lambda c}} = K
$$
\noindent The expected passage time for each car that passes in less than $c$ minutes is $E(T|T<c)$ and by the previous problem
$$
E(T|T<c) = \frac{1}{\lambda} + \frac{c}{1-e^{\lambda c}}
$$
\noindent Hence total wait is that of waiting for $K-1$ cars to pass and then crossing:
$$
\bigg(\frac{1}{e^{-\lambda c}}-1\bigg)\bigg(\frac{1}{\lambda} + \frac{c}{1-e^{\lambda c}}\bigg)= \frac{e^{\lambda c}}{\lambda} - \frac{1 + c\lambda}{\lambda} + c = \frac{e^{\lambda c} - 1}{\lambda}
$$
\end{solution} 
\begin{problem}{2.52(a)} %You can use theorem, exercise, problem, or question here.  Modify x.yz to be whatever number you are proving
How often is the bulb replaced.

\end{problem}

\begin{solution}\ \\

\noindent The janitor replacing the bulb according to when it breaks is a Poisson process with rate $\frac{1}{200}\frac{failures}{day}$. The superposition and the handyman-preventitive-maintenance Poisson process with rate $\frac{1}{100}\frac{replacements}{day}$ is again a Poisson process with rate $\frac{1}{200}+\frac{1}{100} = \frac{3}{200}$ which implies that the lightbulb is changed once every $\frac{200}{3} = 66\frac{2}{3}$ days. 
\end{solution}

\begin{problem}{2.52(b)} %You can use theorem, exercise, problem, or question here.  Modify x.yz to be whatever number you are proving
In the long run what fraction of replacements is due to failure?

\end{problem}

\begin{solution}\ \\

\noindent The rate of replacement due to failure is the relative rate $\frac{.005}{.005+.01} = \frac{1}{3}$.

\end{solution}


\begin{problem}{2.58(a)} %You can use theorem, exercise, problem, or question here.  Modify x.yz to be whatever number you are proving

Compute $\mathbb{P}(N(2)=5)$.

\end{problem}

\begin{solution}\ \\
$$\mathbb{P}(N(2)=5) = \mathbb{P}(N(0+2)-N(0)=5) = \frac{e^{-2\times 2}(2\times 2)^5}{5!} \sim 0.156$$

\end{solution}

\begin{problem}{2.58(b)} %You can use theorem, exercise, problem, or question here.  Modify x.yz to be whatever number you are proving

Compute $\mathbb{P}(N(5)=8|N(2)=3)$.

\end{problem}

\begin{solution}\ \\
$$\mathbb{P}(N(5)=8|N(2)=3) = \mathbb{P}(N(3)-N(0)=5)$$
\noindent by the memoryless property of the Poisson distribution.
$$ \mathbb{P}(N(0+3)-N(0)=5) =  \frac{e^{-2\times 3}(2\times 3)^5}{5!} \sim 0.161$$

\end{solution}

\begin{problem}{2.58(c)} %You can use theorem, exercise, problem, or question here.  Modify x.yz to be whatever number you are proving

Compute $\mathbb{P}(N(2)=3|N(5)=8)$.

\end{problem}

\begin{solution}\ 
\begin{align*}
\mathbb{P}(N(2)=3|N(5)=8) &= \binom{8}{3}\bigg(\frac{2}{5}\bigg)^{3} \bigg(1-\frac{2}{5}\bigg)^{8-3}\\
&\sim 0.279
\end{align*} 
\end{solution}
 
\end{document}
\begin{align*}
\sum_{i=1}^{k+1}i & = \left(\sum_{i=1}^{k}i\right) +(k+1)\\ 
& = \frac{k(k+1)}{2}+k+1 & (\text{by inductive hypothesis})\\
& = \frac{k(k+1)+2(k+1)}{2}\\
& = \frac{(k+1)(k+2)}{2}\\
& = \frac{(k+1)((k+1)+1)}{2}.
\end{align*}
