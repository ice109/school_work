% --------------------------------------------------------------
% This is all preamble stuff that you don't have to worry about.
% Head down to where it says "Start here"
% --------------------------------------------------------------
 
\documentclass[12pt]{article}
 
\usepackage[margin=1in]{geometry} 
\usepackage{amsmath,amsthm,amssymb}
\usepackage{graphicx}

\newcommand{\N}{\mathbb{N}}
\newcommand{\Z}{\mathbb{Z}}
 
\newenvironment{theorem}[2][Theorem]{\begin{trivlist}
\item[\hskip \labelsep {\bfseries #1}\hskip \labelsep {\bfseries #2.}]}{\end{trivlist}}
\newenvironment{lemma}[2][Lemma]{\begin{trivlist}
\item[\hskip \labelsep {\bfseries #1}\hskip \labelsep {\bfseries #2.}]}{\end{trivlist}}
\newenvironment{exercise}[2][Exercise]{\begin{trivlist}
\item[\hskip \labelsep {\bfseries #1}\hskip \labelsep {\bfseries #2.}]}{\end{trivlist}}
\newenvironment{problem}[2][Problem]{\begin{trivlist}
\item[\hskip \labelsep {\bfseries #1}\hskip \labelsep {\bfseries #2.}]}{\end{trivlist}}
\newenvironment{question}[2][Question]{\begin{trivlist}
\item[\hskip \labelsep {\bfseries #1}\hskip \labelsep {\bfseries #2.}]}{\end{trivlist}}
\newenvironment{corollary}[2][Corollary]{\begin{trivlist}
\item[\hskip \labelsep {\bfseries #1}\hskip \labelsep {\bfseries #2.}]}{\end{trivlist}}
\newenvironment{solution}
  {\begin{proof}[Solution]\renewcommand{\qedsymbol}{}}
  {\end{proof}}

\usepackage{listings}
\usepackage{color}

\definecolor{mygreen}{rgb}{0,0.6,0}
\definecolor{mygray}{rgb}{0.5,0.5,0.5}
\definecolor{mymauve}{rgb}{0.58,0,0.82}

\lstset{ %
  backgroundcolor=\color{white},   % choose the background color; you must add \usepackage{color} or \usepackage{xcolor}
  basicstyle=\footnotesize,        % the size of the fonts that are used for the code
  breakatwhitespace=false,         % sets if automatic breaks should only happen at whitespace
  breaklines=true,                 % sets automatic line breaking
  captionpos=b,                    % sets the caption-position to bottom
  commentstyle=\color{mygreen},    % comment style
  deletekeywords={...},            % if you want to delete keywords from the given language
  escapeinside={\%*}{*)},          % if you want to add LaTeX within your code
  extendedchars=true,              % lets you use non-ASCII characters; for 8-bits encodings only, does not work with UTF-8
  frame=single,                    % adds a frame around the code
  keepspaces=true,                 % keeps spaces in text, useful for keeping indentation of code (possibly needs columns=flexible)
  keywordstyle=\color{blue},       % keyword style
  language=Mathematica,                 % the language of the code
  morekeywords={*,...},            % if you want to add more keywords to the set
  numbers=left,                    % where to put the line-numbers; possible values are (none, left, right)
  numbersep=5pt,                   % how far the line-numbers are from the code
  numberstyle=\tiny\color{mygray}, % the style that is used for the line-numbers
  rulecolor=\color{black},         % if not set, the frame-color may be changed on line-breaks within not-black text (e.g. comments (green here))
  showspaces=false,                % show spaces everywhere adding particular underscores; it overrides 'showstringspaces'
  showstringspaces=false,          % underline spaces within strings only
  showtabs=false,                  % show tabs within strings adding particular underscores
  stepnumber=1,                    % the step between two line-numbers. If it's 1, each line will be numbered
  stringstyle=\color{mymauve},     % string literal style
  tabsize=2,                       % sets default tabsize to 2 spaces
  title=\lstname                   % show the filename of files included with \lstinputlisting; also try caption instead of title
}

\begin{document}


 
% --------------------------------------------------------------
%                         Start here
% --------------------------------------------------------------
 
\title{Homework 6}%replace X with the appropriate number
\author{Maksim Levental\\ %replace with your name
MAP 4102} %if necessary, replace with your course title
 
\maketitle
 
\begin{problem}{3.8(a)} %You can use theorem, exercise, problem, or question here.  Modify x.yz to be whatever number you are proving
Find the probability the counter is locked at time $t$.
\end{problem}
 
\begin{solution}\ \\
Using theorem 3.4 in Durrett the probability is 

$$ \frac{\tau}{\tau + 1/\lambda} $$
\end{solution}
\begin{problem}{3.8(b)} %You can use theorem, exercise, problem, or question here.  Modify x.yz to be whatever number you are proving
Compute the limiting fraction of particles that get registered.
\end{problem}
 
\begin{solution}\ \\
Using theorem 3.4 in Durrett the limiting fraction is

$$ \frac{1/\lambda}{\tau + 1/\lambda} $$
\end{solution}
\begin{problem}{3.21}
What is the probability you get a ticket?
\end{problem}
\begin{solution}\ \\
The probability that you get a ticket given that you park for less than 2 hours is zero. The probability that you get a ticket given that you park for a time greater than 2 hours is the probability that you park for greater than 2 hours, $\frac{1}{2}$, times the probability the parking official catches you. This is tantamount to the parking official first arriving earlier than 2 hours before you leave. This is equal to probability of you parking for greater than 2 hours (draw a picture). Hence the probability of you getting ticketed is $\frac{1}{4}$.
\end{solution}

\begin{problem}{4.2(a)}
Write the matrix for the transition rates $Q_{ij}$ and find the stationary distribution.
\end{problem}
 
\begin{solution}\ \\ 
$$
\begin{pmatrix}
-2 & 2 & 0 & 0\\ 
0 & -2 & 2 & 0\\ 
2 & 0 & -4 & 2\\ 
0 & 2 & -2 & 0
\end{pmatrix}
$$

\noindent Using Mathematica Solve[] the stationary distribution $\pi$ is $(0,\frac{1}{2},0,\frac{1}{2})$.
\end{solution} 
\begin{problem}{4.2(b)} 
At what rate does the store make sales?
\end{problem}

\begin{solution}\ \\
50\% of the time the store is in the state with zero computers hence 100\%-50\%=50\% of the time they are in a state of selling computers. In the one and only such state they sell computers at a rate of two per week.
\end{solution}

\begin{problem}{4.9}
Formulate a Markov chain representation and find the long run fraction that the molecule is in each state.
\end{problem}

\begin{solution}\ \\
The transition rate matrix is 
$$
\begin{pmatrix}
-3 & 3 & 0\\ 
 1& -3 & 2\\ 
 0& 4 & -4
\end{pmatrix}
$$

\noindent Using Mathematica Solve[] the stationary distribution $\pi$ is $(\frac{2}{11},\frac{6}{11},\frac{3}{11})$.
\end{solution}
 
\end{document}
\begin{align*}
\sum_{i=1}^{k+1}i & = \left(\sum_{i=1}^{k}i\right) +(k+1)\\ 
& = \frac{k(k+1)}{2}+k+1 & (\text{by inductive hypothesis})\\
& = \frac{k(k+1)+2(k+1)}{2}\\
& = \frac{(k+1)(k+2)}{2}\\
& = \frac{(k+1)((k+1)+1)}{2}.
\end{align*}
