% --------------------------------------------------------------
% This is all preamble stuff that you don't have to worry about.
% Head down to where it says "Start here"
% --------------------------------------------------------------
 
\documentclass[12pt]{article}
 
\usepackage[margin=1in]{geometry} 
\usepackage{amsmath,amsthm,amssymb}
\usepackage{graphicx}
\usepackage{xfrac}

\newcommand{\N}{\mathbb{N}}
\newcommand{\Z}{\mathbb{Z}}
 
\newenvironment{theorem}[2][Theorem]{\begin{trivlist}
\item[\hskip \labelsep {\bfseries #1}\hskip \labelsep {\bfseries #2.}]}{\end{trivlist}}
\newenvironment{lemma}[2][Lemma]{\begin{trivlist}
\item[\hskip \labelsep {\bfseries #1}\hskip \labelsep {\bfseries #2.}]}{\end{trivlist}}
\newenvironment{exercise}[2][Exercise]{\begin{trivlist}
\item[\hskip \labelsep {\bfseries #1}\hskip \labelsep {\bfseries #2.}]}{\end{trivlist}}
\newenvironment{problem}[2][Problem]{\begin{trivlist}
\item[\hskip \labelsep {\bfseries #1}\hskip \labelsep {\bfseries #2.}]}{\end{trivlist}}
\newenvironment{question}[2][Question]{\begin{trivlist}
\item[\hskip \labelsep {\bfseries #1}\hskip \labelsep {\bfseries #2.}]}{\end{trivlist}}
\newenvironment{corollary}[2][Corollary]{\begin{trivlist}
\item[\hskip \labelsep {\bfseries #1}\hskip \labelsep {\bfseries #2.}]}{\end{trivlist}}
\newenvironment{solution}
  {\begin{proof}[Solution]\renewcommand{\qedsymbol}{}}
  {\end{proof}}

\usepackage{listings}
\usepackage{color}
\setlength\parindent{0pt}
\definecolor{mygreen}{rgb}{0,0.6,0}
\definecolor{mygray}{rgb}{0.5,0.5,0.5}
\definecolor{mymauve}{rgb}{0.58,0,0.82}

\lstset{ %
  backgroundcolor=\color{white},   % choose the background color; you must add \usepackage{color} or \usepackage{xcolor}
  basicstyle=\footnotesize,        % the size of the fonts that are used for the code
  breakatwhitespace=false,         % sets if automatic breaks should only happen at whitespace
  breaklines=true,                 % sets automatic line breaking
  captionpos=b,                    % sets the caption-position to bottom
  commentstyle=\color{mygreen},    % comment style
  deletekeywords={...},            % if you want to delete keywords from the given language
  escapeinside={\%*}{*)},          % if you want to add LaTeX within your code
  extendedchars=true,              % lets you use non-ASCII characters; for 8-bits encodings only, does not work with UTF-8
  frame=single,                    % adds a frame around the code
  keepspaces=true,                 % keeps spaces in text, useful for keeping indentation of code (possibly needs columns=flexible)
  keywordstyle=\color{blue},       % keyword style
  language=Mathematica,                 % the language of the code
  morekeywords={*,...},            % if you want to add more keywords to the set
  numbers=left,                    % where to put the line-numbers; possible values are (none, left, right)
  numbersep=5pt,                   % how far the line-numbers are from the code
  numberstyle=\tiny\color{mygray}, % the style that is used for the line-numbers
  rulecolor=\color{black},         % if not set, the frame-color may be changed on line-breaks within not-black text (e.g. comments (green here))
  showspaces=false,                % show spaces everywhere adding particular underscores; it overrides 'showstringspaces'
  showstringspaces=false,          % underline spaces within strings only
  showtabs=false,                  % show tabs within strings adding particular underscores
  stepnumber=1,                    % the step between two line-numbers. If it's 1, each line will be numbered
  stringstyle=\color{mymauve},     % string literal style
  tabsize=2,                       % sets default tabsize to 2 spaces
  title=\lstname                   % show the filename of files included with \lstinputlisting; also try caption instead of title
}

\begin{document}


 
% --------------------------------------------------------------
%                         Start here
% --------------------------------------------------------------
 
\title{Final}%replace X with the appropriate number
\author{Maksim Levental\\ %replace with your name
MAP 4102} %if necessary, replace with your course title
 
\maketitle
 
\begin{problem}{1(a)}
Compute the transition probabilities for this Markov chain.
\end{problem}
 
\begin{solution}\ \\
$$
\begin{pmatrix}
\sfrac{2}{5} & \sfrac{3}{5} & 0 & 0 \\
\sfrac{3}{25} & \sfrac{2}{25}+\sfrac{12}{25} & \sfrac{8}{25} & 0 \\
0 & \sfrac{8}{25} & \sfrac{2}{25}+\sfrac{12}{25} & \sfrac{3}{25} \\
0 & 0 & \sfrac{3}{5} & \sfrac{2}{5} 
\end{pmatrix}
$$

\end{solution}
\begin{problem}{1(b)}
Argue whether this chain has a stationary distribution $\pi$.
\end{problem}

\begin{solution}\ \\

The Markov chain is finite, irreducible, and every state has period 1. Hence there exists a stationary distribution. MATLAB computes $\pi = (\sfrac{1}{12} ,   \sfrac{5}{12} ,   \sfrac{5}{12}  ,  \sfrac{1}{12})$

\end{solution}

\begin{problem}{1(c)}

In the long run, how many purple balls should we expect to see in the first urn?

\end{problem}

\begin{solution}\ \\
$$
\underset{i}{\operatorname{arg\,max}} \, \pi(i) = 1 \wedge 2
$$ 

hence we're most often, and equally likely, to find 1 or 2 balls in the first urn.

\end{solution}

\begin{problem}{1(d)}
Compute the hitting time for state 3.
\end{problem}
 
\begin{solution}\ \\ 

The reduced transition matrix

$$ \tilde{P} =
\begin{pmatrix}
\sfrac{2}{5} & \sfrac{3}{5} & 0 \\
\sfrac{3}{25} & \sfrac{2}{25}+\sfrac{12}{25} & \sfrac{8}{25} \\
0 & \sfrac{8}{25} & \sfrac{2}{25}+\sfrac{12}{25} \\
\end{pmatrix}
$$

Solving $(I-\tilde{P})\vec{u}=\vec{1}$ for $\vec{u}$ yields $\vec{u} = (\sfrac{95}{4},\sfrac{265}{12},\sfrac{55}{3})$ hence it will take on average $\sfrac{95}{3} = 23.75$ turns before there are 3 purple balls in the first urn.


\end{solution} 

\begin{problem}{1(e)} 
Is $X_n$ a martingale? If so, does the MCT apply?
\end{problem}
\begin{solution}\ \\

Yes because the expectation of the fraction of purple balls converges to the ensemble average by the ergodic theorem (so $\mathbb{E}[M_{n+1}]=M_n$). Furthermore the MCT applies because the fraction of purple balls is bounded above and below.

\end{solution}

\begin{problem}{2(a)} 
What is the stationary distribution and what fraction of the time will all chairs be full?
\end{problem}
\begin{solution}\ 
\begin{align*}
q(i,i-1)= 4 &~~ \text{for}~ i = 1,2,3,4 \\
q(i,i+1)= 5 &~~ \text{for}~ i = 0,1,2,3
\end{align*} 

The detailed balance conditions say $5\pi(i-1) = 4\pi(i)$. Setting $\pi(0)= c$ and solving, we have

$$
\pi(1) = \frac{5}{4}c ~~ \pi(2) = \frac{25}{16}c ~~ \pi(3) = \frac{125}{48}c ~~ \pi(4) = \frac{625}{192}c 
$$

The sum of the $\pi$’s is $\sfrac{2101}{256}$ so $c=\sfrac{256}{2101}$ and
$$
\pi(0)=\frac{256}{2101} ~~ \pi(1) = \frac{320}{2101} ~~ \pi(2) = \frac{400}{2101} ~~ \pi(3) = \frac{500}{2101} ~~ \pi(4) = \frac{625}{2101}
$$
From this we see that $\sfrac{625}{2101} = .297$ of the time all four of the chairs are full.
\end{solution}

\begin{problem}{2(b)} 
What fraction of the time is the barber losing business because there are no empty seats?
\end{problem}
\begin{solution}\ \\

From part (a) we see that $29.7\%$ of the time he's turning away potential customers hence $29.7\%$ of his potential business is lost.
\end{solution}

\begin{problem}{2(c)} 
In the long run, how many customers does the barber serve per hour?
\end{problem}
\begin{solution}\ \\

By the ergodic theorem $\sum_{i=0}^4 i\cdot \pi(i)$ is the average number of customers in the barber's shop and hence the average number of customers being served: $\sum_{i=0}^4 i\cdot \pi(i) = 2.44 $

\end{solution}

\begin{problem}{2(d)} 
How long does the average customer wait?
\end{problem}
\begin{solution}\ \\

Using Little's formula the average waiting time $W$ is the average number of customers in the shop, part (c), divided by long-run average rate at which customers arrive and are able to get any seat at all $\lambda(\pi(0)+\pi(1)+\pi(2)+\pi(3))$. Hence
\begin{align*}
W &= \frac{2.44}{5\bigg(\frac{256}{2101} + \frac{320}{2101} + \frac{400}{2101} +  \frac{500}{2101}\bigg)} \\
&= .3424 ~\text{hours}
\end{align*}

\end{solution}

\begin{problem}{3(a)}

Compute $\mathbb{P}({\tau > \sigma})$

\end{problem}
\begin{solution}\ \\

By Eqn. 2.9 in Durrett 

$$
\mathbb{P}({\sigma<\tau}) = \frac{3}{2+3} = \frac{3}{5}
$$

\end{solution}

\begin{problem}{3(b)}

Compute $\mathbb{P}(\tau > \sigma \,|\, \sigma >5)$

\end{problem}
\begin{solution}\ \\

By Eqn. 2.9 in Durrett 

\begin{align*}
\mathbb{P}(\tau > \sigma \,|\, \sigma >5) &= \frac{\mathbb{P}(\tau > \sigma, \sigma >5)}{\mathbb{P}(\sigma > 5)} \\
&= \frac{\int_5^\infty 3 e^{-5s}ds}{1-e^{-3\cdot5}} \\
&= \frac{\frac{3}{5 e^25}}{1-e^{-15}} \\
& = \frac{3}{5 e^{10}(e^{15}-1)}
\end{align*}

\end{solution}

\begin{problem}{3(c)}
Compute $\mathbb{E}[ M ]$.
\end{problem}
\begin{solution}\ \\

By Durrett Chapter 2 summary

$$
\mathbb{E}[ M ] = \frac{1}{3}+\frac{1}{2} - \frac{1}{2+3} = \frac{1}{3} + \frac{1}{2} - \frac{1}{5} = \frac{19}{30}
$$
\end{solution}

\begin{problem}{3(d)}
Compute $\mathbb{E}[S\, | \, m ]$.
\end{problem}
\begin{solution}\ \\

$$
S = \tau + \sigma = M + m  
$$

hence

\begin{align*}
\mathbb{E}[S\, | \, m ] =& \, \mathbb{E}[M + m \, | \, m ]  \\
=& \,\mathbb{E}[M \, | \, m ] + \mathbb{E}[ m \, | \, m ] \\
=&\, \mathbb{E}[M \, | \, m ] + m \\
=&\, \mathbb{E}[\sigma]\bigg( \mathbb{P}(M=\sigma \, | \,m=\sigma) + \mathbb{P} (M=\sigma \, | \,m=\tau) \bigg)+ \\
&\, \mathbb{E}[\tau]\bigg( \mathbb{P} (M=\tau \, | \,m=\sigma) + \mathbb{P} (M=\tau \, | \,m=\tau) \bigg) + m \\
=&\, \sfrac{1}{3}(0+1)+\sfrac{1}{2}(1+0) + m \\
=&\, \frac{5}{6} + m 
\end{align*}
\end{solution}

\begin{problem}{4(a)}
Compute $\mathbb{E}[N(1) \, | \, N(2)]$.
\end{problem}
\begin{solution}\ \\

By Durrett Thm. 2.15 

$$\mathbb{P}(N(1) = m \, | \, N(2) = n ) = {n \choose m} \bigg(\frac{1}{2}\bigg)^m\bigg(1-\frac{1}{2}\bigg)^{n-m} = {n \choose m} \bigg(\frac{1}{2}\bigg)^n
$$  

Let $\eta = N(2)$ then

$$\mathbb{E}[N(1) \, | \, N(2)] =\sum_{i=0}^\eta i\cdot {\eta \choose i} \bigg(\frac{1}{2}\bigg)^\eta = \frac{\eta}{2} = \frac{N(2)}{2}
$$

\end{solution}

\begin{problem}{4(b)}
Compute $\mathbb{E}[N(2) \, | \, N(1)]$.
\end{problem}
\begin{solution}\ \\

$$N(2) =\bigg(N(1+1) - N(1)\bigg) + N(1)$$

Therefore

$$\mathbb{E}[N(2) \, | \, N(1)] = \mathbb{E}\bigg[\bigg(N(1+1) - N(1)\bigg) + N(1) \, \bigg| \, N(1)\bigg] =  \mathbb{E}\bigg[\bigg(N(1+1) - N(1)\bigg) \, \bigg| \, N(1)\bigg] + N(1)$$

But by Durrett Lemma 2.5 $N(1+1) - N(1)$ is independent of $N(r)$ for $0 \leq r \leq 1$ and furthermore distributed $Poisson(\lambda \cdot 1)$. Let $\eta =N(1+1) - N(1)$ with $\eta \sim Poisson(\lambda)$.  Hence $\mathbb{E}[N(2) \, | \, N(1)] = \eta + N(1)$.
\end{solution}

\begin{problem}{5(a)}
Show that $W_n$ is a martingale.
\end{problem}
\begin{solution}\ \\

\begin{align*}
\mathbb{E}[W_{n+1} \, | \, \mathcal{F}_n] &= \mathbb{E}\bigg[\sum_{k=1}^{n+1} H_k X_k \,\bigg | \, \mathcal{F}_n\bigg] \\
&= \mathbb{E}[H_{k+1}X_{k+1} + W_n | \mathcal{F}_n] \\
&=  \mathbb{E}[H_{k+1}X_{k+1} | \mathcal{F}_n] + W_n \\
&= H_{k+1} \mathbb{E}[X_{k+1} | \mathcal{F}_n] + W_n \\
&= H_{k+1} \cdot 0 + W_n = W_n
\end{align*}

The fourth line follows because $H_i$ is independent of $\mathcal{F}_j \,~ \forall i,j$ and the last line follows since $\mathbb{E}[X_i] = 0$.
\end{solution}

\begin{problem}{5(b)}

Show that $\mathbb{E}[W_n^2] = \sigma^2\sum_{k=1}^n\mathbb{E}[H_k^2]$.

\end{problem}
\begin{solution}\ \\

$$
W_n^2 = \bigg(\sum_{k=1}^nH_k X_k\bigg)^2 
$$

Now by the multinomial theorem 

$$
W_n^2 = \sum_{i_1+i_2+ \cdots i_n = 2} {2 \choose i_1,i_2, \cdots, i_n} \bigg((H_1 X_1)^{i_1}(H_2 X_2)^{i_2} \cdots (H_n X_n)^{i_n}\bigg) 
$$
\\
But because there are only binomial terms in the sum
$$
W_n^2 =  \sum_{i=1}^n(H_i X_i)^2 + 2\sum_{i \neq j}(H_i X_i)(H_j X_j) = \sum_{i=1}^nH_i^2 X_i^2 + 2\sum_{i \neq j}(H_i X_i)(H_j X_j)
$$

Then

\begin{align*}
\mathbb{E}[W_n^2] &= \mathbb{E}\bigg[ \sum_{i=1}^nH_i^2 X_i^2 + 2\sum_{i \neq j}(H_i X_i)(H_j X_j)  \bigg] \\
&= \mathbb{E}\bigg[\sum_{i=1}^nH_i^2 X_i^2\bigg]+\mathbb{E}\bigg[ 2\sum_{i \neq j}(H_i X_i)(H_j X_j)\bigg]
\end{align*}

But by independence

\begin{align*}
\mathbb{E}[W_n^2] &= \sum_{i=1}^n \mathbb{E}[H_i^2] \mathbb{E}[X_i^2] + 2\sum_{i \neq j}\mathbb{E}[H_i X_iH_j]\mathbb{E}[ X_j]
\end{align*}

And by the hypothesis $\mathbb{E}[X_i] = 0$

\begin{align*}
\mathbb{E}[W_n^2] &= \sum_{i=1}^n \mathbb{E}[H_i^2] \mathbb{E}[X_i^2] + 2\sum_{i \neq j}\mathbb{E}[H_i X_iH_j]\cdot 0 \\
&=\sum_{i=1}^n \mathbb{E}[H_i^2] \mathbb{E}[X_i^2] \\
&=\sum_{i=1}^n \mathbb{E}[H_i^2] \sigma^2 \\
&= \sigma^2 \sum_{i=1}^n \mathbb{E}[H_i^2]
\end{align*}

\end{solution}

\begin{problem}{6(a)}

Show that $M_n := S_n^2 -n $ is a martingale.

\end{problem}

\begin{solution}\ \\


Let $H_k = 1$ then by the just prior result, with $\sigma^2 = var(X_i)^2 = 1^2$

$$\mathbb{E}[M_n] = \mathbb{E}[S_n^2 - n] = \mathbb{E}[S_n^2] - n = 1^2\sum_1^n\mathbb{E}[H_k^2] - n
$$

But $\mathbb{E}[H_k^2] = 1$ so $\mathbb{E}[M_n] = 0$. Finally
$$\mathbb{E}[M_{n+1} - M_n | \mathcal{F}] = \mathbb{E}[M_{n+1}| \mathcal{F}] - \mathbb{E}[ M_n | \mathcal{F}] = 0-0 = 0$$.

Hence by definition 5.4 in Durrett $S_n^2$ is a martingale.

\end{solution}

\begin{problem}{6(b)}

Compute $\mathbb{E}[\tau]$.

\end{problem}
\begin{solution}\ \\

Using Example 5.11 in Durett $\mathbb{E}_0[\tau] = -(-N)\cdot N = N^2$.

\end{solution}

\end{document}
\begin{problem}{}

\end{problem}
\begin{solution}\ \\

\end{solution}
